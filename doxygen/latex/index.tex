\hypertarget{index_Yleistä}{}\section{Yleistä}\label{index_Yleistä}
Täällä määritellään Ohjelmoinnin tekniikoiden harjoitustyön opiskelijoille näkyvä osa kurssin toteuttamasta koodista.

Käytännössä tämä tarkoittaa kurssin toteuttaman ja opiskelijoiden toteuttaman osan välisiä rajapintaluokkia, niihin liittyviä poikkeusluokkia ja muita tyyppejä sekä opiskelijoiden toteuttaman käyttöliittymän luomiseen tarvittavia funktioita.\hypertarget{index_Rakenne}{}\section{Rakenne}\label{index_Rakenne}
Olennaista tietoa löytyy ainakin seuraavista paikoista tätä dokumentaatiota (sivun yläreunan palkki)\-:
\begin{DoxyItemize}
\item \href{annotated.html}{\tt Luokat}\-: Luettelo tarjotuista luokista
\item \href{namespaces.html}{\tt Nimiavaruudet}\-: Valitsemalla Nimiavaruuden jäsenet löytyy funktion {\ttfamily luo\-Peli()} dokumentaatio
\item \href{files.html}{\tt Tiedostot}\-: Luettelo opiskelijoiden käytettäviksi tarkoitetuista otsikkotiedostoista 
\end{DoxyItemize}